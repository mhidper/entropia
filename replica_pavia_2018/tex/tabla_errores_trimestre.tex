\begin{table}[h]
\centering
\caption{Estadísticos de Error por Trimestre - Período Completo (1996-2025)}
\label{tab:errores_trimestre}
\begin{tabular}{lccccc}
\toprule
\textbf{Trimestre} & \textbf{EM} & \textbf{EAM} & \textbf{DT} & \textbf{MAD} & \textbf{Observaciones} \\
\midrule
Q1 & -0,0391 & 0,2269 & 0,2275 & 0,1211 & 30 \\
Q2 & -0,0476 & 0,1938 & 0,1992 & 0,1393 & 29 \\
Q3 & -0,0195 & 0,1950 & 0,1997 & 0,1285 & 29 \\
Q4 & -0,0410 & 0,1964 & 0,2004 & 0,1141 & 29 \\
\midrule
\textbf{Promedio} & \textbf{-0,0368} & \textbf{0,2033} & \textbf{0,2069} & \textbf{0,1257} & \textbf{117} \\
\bottomrule
\end{tabular}
\begin{tablenotes}
\footnotesize
\item Nota: EM = Error Medio, EAM = Error Absoluto Medio, DT = Desviación Típica, MAD = Desviación Absoluta Mediana.
\item Los errores están expresados en puntos porcentuales de las tasas de crecimiento interanuales.
\item El patrón Q4 ≈ Q3 < Q1 confirma los hallazgos de Pavia et al. (2017).
\end{tablenotes}
\end{table}